% !TeX program = lualatex
\documentclass{Provadis-Klausur}

% Konfiguration
\newcommand\Veranstaltung{Angewandte \LaTeX{}ologie}
\newcommand\Studiengang{BIN}
% Studiengruppe; sollte, falls nicht gewünscht, als Kommentar verbleiben (und daher undefiniert sein)
%\newcommand\Studiengruppe{}
\newcommand\Semester{WS 2025/2026}
\newcommand\Dozent{Prof. Dr. Eric Hutter}
\newcommand\Datum{\today}
\newcommand\DauerMinuten{60}
% Hilfsmittel; falls undefiniert, wird "Keine" gesetzt
\newcommand\Hilfsmittel{Lua\LaTeX\ oder Xe\LaTeX. Im Notfall geht auch pdf\LaTeX.}

\newcommand\PunkteZusatzspalten{
	{Bonus\footnotemark}/5%,{Noch eine Spalte}/3
}
% kann genutzt werden, um zusätzlichen Text nach der Punkteverteilung zu setzen,
% z.B. eine Fußnote zu einer Zusatzspalte, denn innerhalb der Zusatzspalten funktioniert \footnote aus irgendeinem Grund nicht
\newcommand\NachPunkteverteilung{%
	\vspace{-0.5em}% damit alles noch auf eine Seite passt...
	\footnotetext{Durch Übungspunkte kann ein Bonus von 5 Punkten auf die Klausur erworben werden.}
}% wird nach Punkteverteilungs-Tabelle eingefügt

% Beispiel, wie auf der ersten Seite ein Wasserzeichen (z.B. "Lösung" oder "Probeklausur") eingebunden werden kann.
% Die aktivierte loesung-Option überschreibt das hier gesetzte Wasserzeichen allerdings in jedem Fall.
\DraftwatermarkOptions{stamp=true}
\SetWatermarkText{\textbf{Dokumentation}}
\SetWatermarkScale{0.66}

\begin{document}
\Aufgabe[Grundlagen]{1,2,3}
\begin{enumerate}
	\item Passen Sie die Konfigurations-Makros in der \texttt{.tex}-Datei an, um die Klausur auf Ihre Veranstaltung zuzuschneiden.
	\item Leiten Sie Aufgaben mit \texttt{\textbackslash Aufgabe\{1,2,3\}} ein.
	
		\emph{Hinweis:} Die Semantik des \texttt{\textbackslash Aufgabe}-Befehls ist vergleichbar mit \LaTeX-Standardbefehlen wie \texttt{\textbackslash chapter}, \texttt{\textbackslash section} usw. -- diese sollten in Klausuren aber \textbf{nicht} genutzt werden.
		\begin{enumerate}
			\item Das verpflichtende Argument enthält die mit Komma getrennten Punktzahlen der Teilaufgaben.
			\item Optional kann (wie in dieser Aufgabe) ein Titel der Aufgabe (hier: "`Grundlagen"') vergeben werden: \texttt{\textbackslash Aufgabe[Grundlagen]\{1,2,3\}}
			\item Wenn nur ein einziger Eintrag vorhanden ist, wird im Aufgabenkopf keine Auflistung der Teilpunkte ausgegeben, sondern nur die Gesamtpunktzahl genannt.
		\end{enumerate}
	\item Punkteverteilung und Gesamtpunktzahl auf dem Deckblatt werden automatisch generiert.
		\begin{enumerate}
			\item Wenn sich die Anzahl der Aufgaben und/oder die Verteilung der Punkte auf die Aufgaben ändert, sind zwei bis drei \LaTeX-Durchläufe nötig, um das Deckblatt in einen konsistenten Zustand zu bringen.
			\item Die Klasse prüft am Ende des Dokuments, ob das Deckblatt inkonsistent zu den Aufgaben ist und wirft in einem solchen Fall einen \textbf{Fehler}.
				Das schützt effektiv vor voreilig eingereichten, fehlerhaften Klausuren.
			\item Auch bei Änderungen der Seitenzahl werden zwei \LaTeX-Durchläufe benötigt, bis die Angaben in der Fußzeile korrekt sind.
		\end{enumerate}
\end{enumerate}

\Aufgabe{1}
Kompilieren Sie Ihre ganz persönliche Klausur!

\emph{Hinweis:} Beachten Sie, dass für diese Aufgabe keine Teilaufgaben im Aufgabenkopf verzeichnet sind.
Konkret wurde diese Aufgabe mit \texttt{\textbackslash Aufgabe\{1\}} eingeleitet und enthält somit nur eine einzige "`Teilaufgabe"', der Rest geschieht automatisch.

Die Nummer der Aufgabe (hier: 2) muss nicht explizit angegeben werden und wird automatisch hochgezählt.

\Aufgabe{1}
Bei Bedarf können der Punkteverteilung Zusatzspalten hinzugefügt werden, beispielsweise für Bonuspunkte aus Hausübungen. Beachten Sie hierzu die Beispielimplementierung in der \texttt{.tex}-Datei.

\Aufgabe{5}
Der Vorteil von \LaTeX{} liegt ganz klar im Formelsatz:
\begin{gather*}
	\exp(z) := \sum_{i=0}^\infty \frac{z^k}{k!} \\
	\exp(ix) = \cos x + i\sin x
\end{gather*}
Die Mathematikumgebungen sind so konfiguriert, dass ihr Design den Standardeinstellungen des Word-Formeleditors entspricht (Schriftart \emph{Cambria Math}) -- zumindest unter Lua\LaTeX\ oder Xe\LaTeX; bei Nutzung von pdf\LaTeX\ wird stattdessen \emph{Times New Roman} genutzt.

\emph{Hinweis:} Bei Bedarf können zusätzlich die Pakete \texttt{amsmath}, \texttt{mathtools} etc. geladen werden.

\platz{1cm}
\begin{loesung}
	Diese Lösung wird nur bei aktivierter \texttt{loesung}-Option angezeigt!
\end{loesung}

\Aufgabe[Lösungsdokumente]{42}
Es werden einige Hilfsmakros angeboten, um die Klausur vorzubereiten:
\begin{itemize}
	\item Es kann mit der Klassenoption \texttt{loesung} kompiliert werden, um ein Lösungsdokument zu erhalten.
		Dieses wird mit einem Wasserzeichen auf der ersten Seite entsprechend markiert.
	\item \texttt{\textbackslash platz\{LÄNGE\}} erzeugt einen Leerraum der Größe \texttt{LÄNGE}, der allerdings \emph{nicht} in Lösungsdokumenten vorhanden ist.
	\item Der Inhalt der \texttt{nurAufgabe}-Umgebung wird \textbf{nicht} in Lösungsdokumenten gesetzt.
	\item Der Inhalt der \texttt{nurLoesung}-Umgebung wird \textbf{nur} in Lösungsdokumenten gesetzt.
	\item Der Inhalt der \texttt{loesung}-Umgebung wird \textbf{nur} in Lösungsdokumenten gesetzt und durch einen Kasten mit der Überschrift "`Lösung"' hervorgehoben.
	\item Das \texttt{ARGUMENT} des \texttt{\textbackslash loes\{ARGUMENT\}}-Befehls wird \textbf{nur} in Lösungsdokumenten gesetzt, dafür jedoch \textbf{fett}.
\end{itemize}


\end{document}